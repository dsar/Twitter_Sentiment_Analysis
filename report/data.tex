\section{Data Description}
\label{sec:data_description}
The given dataset is composed of 2.5 million labeled and equally separated tweets that used to contain a positive :) and or negative :( smiley, which is supposed to be used for training, and another $10,000$ unlabeled tweets for testing. 
Given the smiley-filtered test tweets and considering only the remaining text, we are asked to classify them in positive and negative sentiments. 
The fact that training dataset is equally separated and the absense of a neutral sentiment made our problem easier. 

On the other hand, there are some important properties of the dataset that should not be discarded. 
Twitter text is written by regular users in daily life, so in that informal language that is used for tweets, there are usually non existing words which nonetheless makes sense either in the morphological or in the pragmatic layer of a language. 
Moreover, there are plenty of characters that are not used for the formal written language, which is the basis of most pretrained models, but they usually indicate something meaningful in tweets, like punctuation or other types of emotions. 
Finally, the existence of compound words, syntactically incorrect sentences, different conjugations of the same word and many other similar cases are the problems to be attacked by preprocessing steps..