\section{Data Description}
\label{sec:data_description}
In this section we describe the given Twitter dataset. We were given 2.5 million
equally seperated on tweets that used to contain a positive :) and or negative :( smiley and another $10,000$ test unlabled tweets. Given, the smiley-filtered test tweets and considering only the remaining text, we were asked to classify them in positive and negative sentiments. The fact that the tweets are equally seperated and the absense of a neutral sentiment made our problem easier. Also, some preprocessing has been done on the given data i.e. all the users and urls have been changed to <user> and <url> tags respectively and all the tweets has been transformed to lowercase characters.

However some other observations on the data had to be considered. Twitter text is written by regular users, in informal language which most of the time contains non existing words which nonetheless makes sense either in the morphological or in the pragmatic layer of a language. Moreover, there are plenty of characters that indicate something meaningfull like punctuation or other types of emoticons. Finally, the existence of compound words, syntactically incorrect sentences, different conjugations of the same word and many other such problems had to be taken into account. In the following, we will present some preprocessing techniques which will make the feature extraction approaches easier and will boost our accurancy.