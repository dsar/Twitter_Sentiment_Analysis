\section{Representations of Tweets}
\label{sec:tweetRepresentations}

In this section, we present all the algorithms that we investigated for converting our input tweets to numerical representations.
These representations try to capture the semantic meaning of the tweets 
while having a constant size, which is independent of the words used in each tweet.
Thus, letting $N$ be the number of input tweets, we create an $N \times D$ matrix
with each line representing a different tweet with $D$ features.


\paragraph{\textbf{Bag of Words Representation}}
Bag of Words \cite{harris1954distributional} is a very common and straightforward numerical representation of text data.
Given that we have a vocabulary $V$ with finite size, we create
an array of size $|V|$ with $1s$ in the position of the words belonging in this text and $0s$ everywhere else.
Thus, in our case, we would have an $N \times |V|$ matrix representing our input tweets.

The problem with this representation is that $|V|$ can be arbitrary big especially if we do not handle appropriately slang and words with spelling mistakes, which can actually be solved by using the preprocessing methods that we proposed in Section~\ref{sec:preprocessing}. 
Another problem of this representation is that it does not comprise any semantic information about the significance of each word in a tweet. 


\paragraph{\textbf{TF-IDF Representation}}
The significance of the words in tweets is captured by a more sophisticated representation, which uses the Term Frequency - Inverse Document Frequency (TF-IDF).
TF-IDF \cite{sparck1972statistical} works by determining the relative frequency of a word in a specific document compared to the inverse proportion of that word over the entire document corpus.
Words that are common in a single or small group of documents tend to have higher TF-IDF numbers than common words such as articles and prepositions. 

As can be used as a standalone representation, this metric can also be used to weight the Bag of Words representation, in which case the $1s$ in $N \times |V|$  representation matrix are replaced by $TF-IDFs$.


\paragraph{\textbf{N-grams Extension}}
In both of the above representations the $N \times |V|$ matrix
can become even bigger if we add sequences of contiguous words in our vocabulary.
These sequences are called \textit{n-grams} with $n$ being the maximum number of words in a sequence.  
For example, for bigrams, in the worst case the size of the vocabulary becomes $|V| + {|V| \choose 2}$.
As we will see in the experimental evaluation we tested representations with up to 8-grams. 


\paragraph{\textbf{Word Embeddings Representation}}
Word Embeddings (WE) \cite{DBLP:journals/corr/MikolovSCCD13} is a recently proposed representation that maps words to high dimensional vectors of real numbers. 
This representation tries to capture the multiple contexts in which a word can appear in a text.
Hence, it guarantees that words that appear in similar context will have vectors that are very close each other.

The advantage of WE compared to the other aforementioned  representations is that each word is represented by a fixed-size vector regardless of the used vocabulary.
In tweet classification problem, given a of size $|V|$, tweets can be represented by a $|V| \times D$ matrix, where $D$ is the dimension of our vectors.
Depending on the vocabulary, we have different options for WE methods, which are listed below.

\subsubsection{GloVe Training Methods}
\textit{GloVe} \cite{pennington2014glove} has an open-source implementation of related training methods.
As a first option, we used the given training tweet dataset to obtain the representation matrix using both the given implementation and \textit{glove-python}\footnote{\url{http://github.com/maciejkula/glove-python}}, which is the python implementation of the original algorithm that uses \textit{SGD} for training.

\subsubsection{Pre-trained GloVe Embeddings}
As another option, we also employed the pre-trained set of Word Embeddings for tweets, published by GloVe\footnote{\url{http://nlp.stanford.edu/projects/glove/}}.
These vectors were trained with $2$ billions tweets containing $1.2$ millions words.
Regarding the one order of magnitude difference in terms of the amount of training data, their corpus is observed to cover a high percentage of the words used in our given train set.

\subsubsection{Hybrid Method}
Since not all of the words in our training dataset were found in the pre-trained Word Embeddings, we decided to train our own vectors for these remaining words and append them to the pre-trained ones.


\paragraph{\textbf{From Word Embeddings to Tweet Embeddings}}
All aforementioned representation methods enable us to construct vector representations for all the words in the vocabulary used by the tweets ($|V| \times D$ matrix).
Since tweets contain a subset of the words within the vocabulary, it's necessary to aggregate multiple rows of this matrix, in order to create representation for tweets ($N \times D$ matrix).

After investigating with various ways of aggregation (concatenation, summation and averaging), mean of WEs of all the words in a tweet is found to be the most expressive one for our classification algorithms.


\paragraph{\textbf{Paragraph Embeddings}}
One method that constructs directly embeddings for paragraphs (tweets, in our case) is proposed by \cite{le2014distributed}.
The advantage of this method is the fact that the order of the words in a tweet is also taken into account, by permuting them multiple times, which is claimed to reveal the semantic meaning of the phrases in a paragraph/tweet.
There are two model representations described in this paper: Distributed bag of words (DBoW) and Distributed Memory (DM) both of which are included in our experiments.
We observed that DBoW model is more efficient and requires less memory.
